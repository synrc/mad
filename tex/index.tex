\section{MAD: Erlang Containers}

\subsection{Purpose}
We were trying to make something minimalistic that fits out \footahref{https://github.com/synrc}{application stack}.
The main idea of mad is to provide clean and simple rebar-like fast dependency manager that
is able to build several types of packages and provides interface of containered deployments
to virtualiezed environments.

\subsection{Several Types of Packaging}
The key feature of mad is ability to create single-file bundled web sites.
This target escript is ready to run on Windows, Linux and Mac.

\subsection{Deployment Options}
As a deploy tool mad is also supposed to launch, start, stop and manage containers, locally or remote.
You can make containers from different type of packages, like making runc container with beam release.

\subsection{OTP Compliant}
Mad supports ERTS boot files generation with systools and erlang application format used by OTP.
This is the main format of application repository. Also boot files are suported on both LING and BEAM.

\subsection{Tiny Size}
And the good part:

\vspace{1\baselineskip}
\begin{lstlisting}
                      Sources        Binary
    mad               967 LOC        52 KB
    rebar             7717 LOC       181 KB
\end{lstlisting}
\vspace{1\baselineskip}

\subsection{History}

We came to conclusion that no matter how perfect your libraries are,
the comfort and ease come mostly from developing tools.
Everything got started when \footahref{https://github.com/proger}{Vladimir~Kirillov} decided to
replace Rusty's sync beam reloader. As you know sync uses
filesystem polling which is neither energy-efficient nor elegant. Also
sync is only able to recompile separate modules while
common use-case in N2O is to recompile DTL templates
and LESS/SCSS stylesheets. That is why we need to recompile
the whole project. That's the story how \footahref{https://github.com/synrc/active}{active} emerged.
Under the hood active is a client subscriber
of \footahref{https://github.com/synrc/fs}{fs} library, native filesystem listener for Linux, Windows and Mac.

De-facto standard in Erlang world is rebar.
We love rebar interface despite its implementation.
First we plugged rebar into active and then decided to drop its support,
it was slow, especially in cold recompilation.
It was designed to be a stand-alone tool, so it has some
glitches while using as embedded library.
Later we switched to Makefile-based build tool \footahref{https://github.com/synrc/otp.mk}{otp.mk}.

The idea to build rebar replacement was up in the air for a long time.
The best minimal approach was picked up by \footahref{https://github.com/s1n4}{Sina~Samavati},
who implemented the first prototype called 'mad'. Initially mad
was able to compile DTL templates, YECC files, escript (like
bundled in gproc), also it had support for caching with side-effects.
In a month I forked mad and took over the development under the same name.

\vspace{1\baselineskip}
\begin{lstlisting}[caption=Example of building N2O sample]
                                   Cold       Hot
    rebar get-deps compile         53.156s    4.714s
    mad deps compile               54.097s    0.899s
\end{lstlisting}
\vspace{1\baselineskip}

\vspace{1\baselineskip}
\begin{lstlisting}[caption=Example of building Cowboy]
                                   Hot
    make (erlang.mk)               2.588s
    mad compile                    2.521s
\end{lstlisting}
\vspace{1\baselineskip}
