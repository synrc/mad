\section{MAD: Erlang Build and Deploy Tool}

\subsection{History}

We realized no matter how perfect your libraries are,
the comfort and easy come mostly from developing tools.
Everything got started when Vladimir Kirillov decided to
replace Rusty's SYNC beam reloader. As you know SYNC uses
filesystem polling which is neither economic nor elegant. Also
SYNC is able only to reload separate modules while
common use-case in N2O is to recompile DTL templates
and LESS/SCSS stylesheets. That is why we need to trigger
the whole project recompilation. That's the story how ACTIVE
has been emerged. Under the hood ACTIVE is a client subscriber
of FS library, native async filesystem listener
for Linux, Windows and Mac.

De-facto standard in Erlang world is REBAR.
We love REBAR interface despite its implementation.
When we plugged REBAR to ACTIVE and then decided totally
drop REBAR support and where switched to makefile
based build tool OTP.MK. Becase the overall size
of REBAR was bigger then whole our Web Stack.
And of course it is slow, especially in cold recompilation.
It was designed to be a stand-alone tool, so it has some
glitches while using as embedded library.

Our idea to build REBAR replacement was picked up by Sina Samavati,
he implemented the first prototype called MAD. Initially MAD
was able to compile DTL templated, YECC files, ESCRIPT (like
bundled in GPROC), also it was support caching with side-effects.
After month I continue the MAD fork with the same name.

\vspace{1\baselineskip}
\begin{lstlisting}[caption=Example of building N2O sample]
                                   Cold       Hot
    rebar get-deps compile         53.156s    4.714s
    mad deps compile               54.097s    0.899s
\end{lstlisting}
\vspace{1\baselineskip}

\vspace{1\baselineskip}
\begin{lstlisting}[caption=Example of building Cowboy]
                                   Cold
    make (ERLANG.MK)               2.588s
    mad compile                    2.521s
\end{lstlisting}
\vspace{1\baselineskip}

\subsection{Introduction}

We was trying to made something minimalistic that fits out Web Stack.
Besides we wanted to use our knowledge of other build tools like lein, sbt etc.
Also for sure we tried sinan, ebt, makefile based scripts.

Synrc MAD has following commands DSL:

\vspace{1\baselineskip}
\begin{lstlisting}
  BNF:
      invoke := mad params
      params := [] | run params
         run := command [ options ]
         cmd := app | lib | deps | compile | bundle
                run | stop | attach | repl
\end{lstlisting}
\vspace{1\baselineskip}

It seems to us more natural, you can specify random
commands set with different specifiers (options).

\subsection{Single-File Bundling}

The key feature of MAD is ability to create single-file bundled web sites.
Thus making Joe's dream to boot like node.js come true.
This target escript is ready to run on Windows, Linux and Mac.

To make this possible we implemented zip filesytem inside escript.
MAD packages priv directories along with ebin and configs.
Also you can redefine each file in zip fs inside target
escript by creation the copy with same path locally near escript.
After launch all files copied to ETS.
N2O also comes with custom COWBOY static handler that is able to
read static files from this cached ETS filesystem.
Also bundle are compatible with ACTIVE online realoading and recompilation.

\subsection{Templates}

MAD also come with N2O templates. So you can bootstrap N2O site
just having a single copy of MAD.

\vspace{1\baselineskip}
\begin{lstlisting}
    \$ mad app sample
    \$ cd sample
    \$ mad deps compile plan bundle web_app
\end{lstlisting}
\vspace{1\baselineskip}

After that just can run ./web_app under Windows, Linux and
Mac and open \footahref{http://localhost:8000}{http://localhost:8000}.

\vspace{1\baselineskip}
\begin{lstlisting}
    C:\> escript web_app
    Applications: [kernel,stdlib,crypto,cowlib,ranch,cowboy,compiler,syntax_tools,
                   erlydtl,gproc,xmerl,n2o,n2o_sample,fs,active,mad,sh]
    Configuration: [{n2o,[{port,8000},{route,routes}]},
                    {kvs,[{dba,store_mnesia},
                          {schema,[kvs_user,kvs_acl,kvs_feed,kvs_subscription]}]}]
    Erlang/OTP 17 [erts-6.0] [64-bit] [smp:4:4] [async-threads:10] [kernel-poll:false]

    Eshell V6.0  (abort with ^G)
    1>
\end{lstlisting}
\vspace{1\baselineskip}

\subsection{Deploy}

MAD is also supposed to be also a deploy tool with ability to
deploy not only to our resources like EoX, Voxoz (LXC/Xen) but
also to Heroky and others.

\subsection{OTP Compliant}

MAD is respectful to OTP directories. It supports two kinds of directory layouts:

\vspace{1\baselineskip}
\begin{lstlisting}[caption=Solution]
    ├── apps
    ├── deps
    ├── rebar.config
    └── sys.config
\end{lstlisting}
\vspace{1\baselineskip}

\vspace{1\baselineskip}
\begin{lstlisting}[caption=OTP Application]
    ├── deps
    ├── ebin
    ├── include
    ├── priv
    ├── src
    └── rebar.config
\end{lstlisting}
\vspace{1\baselineskip}
